Um die verschiedenen Umsetzungsmöglichkeiten beziehungsweise Optionen evaluieren zu können, waren für uns folgende Kriterien entscheidend: Skalierbarkeit, Preis-/Nutzen-Verhältnis, Erfüllung der Anforderungen von Statistance und Komplexität. Im Folgenden werden auf die einzelnen Evaluierungskriterien eingegangen und näher erläutert.

\subsection*{\textbf{Skalierbarkeit/Erweiterbarkeit}}\label{subsubsec:Skalierbarkeit/Erweiterbarkeit} 
Skalierbarkeit bezeichnet die Fähigkeit eines Systems, Netzwerks oder Prozesses bei steigender Nachfrage zu wachsen und sich an an die veränderten Gegebenheiten anzupassen \cite{skalierbarkeit}. Dabei wird zwischen vertikaler Skalierung (scale up) und horizontaler (scale out) Skalierung unterschieden. Im Gegensatz zur vertikalen Skalierung, wo zusätzliche Ressourcen zu einem Knoten/Rechner hinzugefügt werden, wird bei horizontaler Skalierung die Steigerung der Leistung des Systems durch das Hinzufügen von zusätzlichen Rechner/Knoten erreicht. Bezogen auf die Anforderungen unseres Projektes hatten wir Skalierbarkeit im Wesentlichen wie folgt definiert: Wie einfach lassen sich neue Schnittstellen von Drittsystemen anbinden und wie verhält sich das System bei steigender Nutzung und Nachfrage?

\subsection*{\textbf{Erfüllung der Anforderungen von Statistance}}\label{Erfüllung der Anforderungen von Statistance} 
Unter diesem Evaluierungskriterium haben wir beurteilt, ob und wie weit sich die
die Anforderungen von Statistance (siehe Kapitel \ref{chap: Anforderungsanalyse}) mit der Lösung/Option erfüllen lassen. Wichtig hierbei war, dass die praktischen Bedürfnisse von Statistance erfüllt wurden.

\subsection*{\textbf{Preis/Nutzen}}\label{Preis/Nutzen}
Hierbei untersuchten wir, wie hoch die variablen sowie die fixen Kosten waren, um die Lösung zu betreiben. Dabei wurde bewertet, wie gut das Verhältnis zwischen den Kosten und Nutzen für Statistance war. Hierbei war zu beachten, dass die davor genannten Anforderungen (siehe Kapitel \ref{chap: Anforderungsanalyse}) dem Nutzen entsprachen und eine Übererfüllung der Anforderungen nicht automatisch zu einer besseren Bewertung geführt hat. 

\subsection*{\textbf{Komplexität}}\label{Komplexität}
Hierunter bewerteten wir die Komplexität des Gesamtsystems und den allgemeinen Aufwand, um die benötigte Infrastruktur aufzubauen und die Lösung zu betreiben. Dabei berücksichtigten wir auch den Einarbeitungsaufwand in die Lösung, um anschließend den Betrieb und Wartung der Lösung gewährleisten zu können.
