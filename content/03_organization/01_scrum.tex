\section{SCRUM}
Im Folgenden wird die Entscheidung für ein agiles Vorgehensmodell begründet sowie die genaue Rollen-und Aufgabenverteilung dargestellt.

\subsection{Agiles Vorgehen}
Prozessmodelle legen den organisatorischen Rahmen für eine Entwicklung fest indem sie unter anderem den Arbeitsablauf, die dabei durchzuführenden Aktivitäten sowie die Verantwortlichkeiten und Kompetenzen im Team festlegen. \cite{amberg2011wertschopfungsorientierte} Prozessmodelle ermöglichen damit eine kontrollierte und disziplinierte Entwicklung. \cite{amberg2011wertschopfungsorientierte} Zu Beginn des Projektes wurden grundlegende Prozessmodelle, wie zum Beispiel das Wasserfallmodell ausgeschlossen, da diese insbesondere für das vorliegende Projekt einige Nachteile aufweisen. Häufig können bei dieser Art von Prozessmodellen Risikofaktoren weniger berücksichtigt werden. Zudem ist das sequentielle vollständige Durchlaufen einer Projektphase für das vorliegende Projekt nicht immer sinnvoll. \cite{amberg2011wertschopfungsorientierte} Code-getriebene Modelle, wie beispielsweise das evolutionäre Modell, hemmen den Entwicklungsprozess und mögliche Wege durch festgelegte Zwischenergebnisse. \cite{amberg2011wertschopfungsorientierte, balzert2000lehrbuch} Agile Entwicklungsmodelle wie Scrum setzen auf die Selbstorganisation der einzelnen Teammitglieder. \cite{scrum} Durch Festlegung von Meetings, Artefakten und Rollen bietet das Modell einen groben Rahmen, lässt aber genug Freiraum, um Anforderungen immer wieder kontinuierlich an die Gegebenheiten und die Kundenwünsche anzupassen.
Aufgrund der zuvor genannten Aspekte, der Teamgröße und der Entwicklung eines Produktes, welches genau den Anforderungen von Statistance entsprechen soll, wurde Scrum als agiles Vorgehensmodell gewählt. Durch regelmäßige Rücksprachen und das demonstrieren verschiedener Prototypen konnten die Anforderungen kontinuierlich angepasst werden. Zum Projektstart wurden zweiwöchige Sprints festgelegt. Eine Ausnahme stellt der erste Sprint dar, welcher eine Woche umfasste. In der ersten Woche wurden insbesondere organisatorische Aufgaben abgewickelt. Zweimal wöchentlich gab es Meetings vor Ort und via Skype. Die Sprint Planning Meetings fanden zu Beginn eines Sprints statt. Bei den Meetings wurden Inhalte des nächsten Sprints festgelegt und dokumentiert. Hierbei wurden Priorität und Aufwand der Aufgaben mit einbezogen. Aufgaben aus dem Product Backlog wurden ausgewählt und entsprechenden Teammitgliedern zugewiesen. Mit der Festlegung der Aufgaben wurde ebenfalls ein Sprintziel definiert. Rücksprachen mit Statistance fanden regelmäßig ca. alle zwei Wochen statt. Zwischen den Treffen vor Ort gab es zudem regelmäßige Rücksprachen über Kommunikations-Tools, welche in Abschnitt \ref{subsubsec:kommunikation} beschrieben werden. Daily Scrum Meetings wurden nicht abgehalten, jedoch regelmäßig Updates während eines Sprints durch einzelne Teammitglieder gegeben beziehungsweise eingeholt. Das wöchentliche Skype-Meeting diente als eine Art Sprint Review Meeting beziehungsweise Retrospektive. Die erzielten (Zwischen-) Ergebnisse wurden nacheinander besprochen. Jeder Teilnehmer des Meetings konnte seine (Zwischen-) Ergebnisse präsentieren und Feedback zu anderen geben. Zukünftige Verbesserungen und die Weiterentwicklung wurden ebenfalls besprochen. Durch das Einholen von Feedback durch Statistance konnten neue Anforderungen nach jeder Iteration in den nächsten Sprint einfließen. Ab der Entwicklungsphase ging aus jedem Sprint ein lauffähiges Produkt hervor, welches sich immer mehr dem reinen Endprodukt annäherte. 


\subsection{Rollen- und Aufgabenverteilung}
An dem \textit{Praxisprojekt Anwendungssysteme} haben insgesamt sechs Personen mitgewirkt. Die Mitglieder studieren im Bachelor oder Master Wirtschaftsinformatik beziehungsweise Information Systems Management oder Informatik. Es vereinen sich daher unter anderem Kompetenzen aus den Bereichen Frontend- und Backendentwicklung sowie Projektmanagement. Jedes Mitglied konnte seine Kompetenzen bei entsprechenden Aufgaben gut einbringen, die eigene Kompetenzen ausbauen und neue dazu gewinnen.
Die einzelnen Aufgaben jedes Teammitglieds sind in Tabelle \ref{tab:tasks} dargestellt.
\begin{table}[h!]
\begin{tabular}{|p{7cm}|c|c|c|c|c|c|}
\hline
\textbf{A} & \textbf{F}  & \textbf{J} & \textbf{J}& \textbf{T} & \textbf{M} & \textbf{K} \\ \hline \bottomrule
Projektmanagement und \newline Architekturüberblick &  &  &  & X &  & X \\ \hline
Recherche & X & X & X & X & X & X \\ \hline
Batch-Jobs &  &  &  & X &  &  \\ \hline
Sage 100 Integration &  & X & X &  &  &  \\ \hline
API & X &  & X & X &  & X \\ \hline
Spring Security &  &  &  & & X &  \\ \hline
Frontend  & X &  & X &  &  &  \\ \hline
Config Management  &  & X &  & X & X &  \\ \hline
API Gateway &  & X &  &  & X & X \\ \hline
\end{tabular}
\caption{Aufgabenverteilung}
\label{tab:tasks}
\end{table}

Für die Kommunikation mit Statistance und M. wurde ein Projektleiter festgelegt. Der Projektleiter hatte zudem die Aufgabe, dass Abgaben (zum Beispiel Zwischenpräsentationen) fristgerecht fertiggestellt und abgegeben werden. Eine weitere Person im Team hatte die Aufgabe die technische Umsetzung der Architektur im Blick zu behalten, damit alle notwendigen technischen Aufgaben richtig und fristgerecht umgesetzt werden. An der Entwicklung waren alle sechs Teammitglieder beteiligt, sodass sich das Entwicklungsteam aus allen sechs Teammitgliedern zusammensetzte.