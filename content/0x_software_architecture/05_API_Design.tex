\namedsection{API Design}{K}
In den nachfolgenden Abschnitten werden sowohl das Vorhandensein der API begründet, die Umsetzung und der Aufbau der API beschrieben als auch die konkreten Endpoints dargestellt.

\subsection{Aufbau REST API}\label{subsubsec:RestAPI}
Die entwickelte API ermöglicht mittels HTTP-Requests (GET, PUT, POST, DELETE) das Zugreifen auf Informationen. Sie dient als Schnittstelle zwischen der Statistance-Applikation und dem ERP-System des Kunden. Insbesondere Informationen zu erhaltenen Lieferungen mit den entsprechenden Produkten sowie Informationen über Hersteller und Lieferanten werden von der Statistance- Applikation benötigt. Die konkreten Daten sollen in unterschiedlichen Zeitabständen abgerufen werden können. Ziel ist hierbei, dass die Statistance-Applikation die aktuellen Daten weiterverarbeitet. Aus diesem Grund müssen verschiedene Batch-Jobs ausgeführt und gesteuert werden können. Für die Umsetzung wurde Spring Boot verwendet. \cite{springboot} Um einen Zugang zur Applikation zu ermöglichen sind verschiedene REST-Endpoints notwendig. Daher wurden zunächst sogenannte Controller-Klassen erstellt. Für sechs verschiedene Entitäten (delivery, order, product, supplier, manufacturer, employee) sowie für den Scheduler (Scheduling der Batch-Jobs), die Batch-Jobs und das Batch-Job-Setting wurde je ein Controller implementiert. Controller dienen der Verarbeitung eingehender HTTP-Requests und ermöglichen die Rückgabe eines passenden Response. Die entsprechende Annotation \textit{@RESTController} teilt der Spring Boot Applikation mit, dass eingehende HTTP-Request von dieser Klasse behandelt werden. 
Für die Rückgabe bei den Controllern der Entitäten wurden sogenannte \textit{ResponseEntity} verwendet. Diese repräsentieren den gesamten HTTP-Response (Statuscode, Header, Body) und wurden verwendet, um beim Auftreten von Verarbeitungsfehlern des Requests, fehlerbezogene Informationen an den Nutzer zurück zu geben. \cite{springbaeldung} Für jede Entität wurde außerdem eine Klasse mit den entsprechenden Attributen, eine Service-Klasse und ein Repository Interface implementiert.
Repository Interfaces (erben von Repository) ermöglichen die Erkennung einzelner Klassen als Komponenten und ermöglichen den Zugriff auf die Datenbank. 
Die \textit{@Service} Annotation in der jeweiligen Service-Klasse wird verwendet, um die Geschäftslogik in einem anderen Layer, getrennt von der \textit{@RESTController} Klasse zu schreiben. \cite{springtutorialspoint} 
Die letzte Komponente ist die entsprechende Klasse, welche die jeweilige Entität, den Scheduler, den Batch-Job oder das Batch-Job-Setting selbst darstellt. Sie enthält die notwendigen Attribute der Ressource und dient damit als Repräsentation dieser.
\subsection{API Dokumentation}
Für die API Dokumentation wurde Swagger verwendet. Mit der Swagger UI können die Ressourcen der API visualisiert und mit ihnen interagiert werden, ohne das die Implementierungslogik vorhanden ist. \cite{swagger} Letztere wird automatisch aus der OpenAPI-Spezifikation generiert. Die Swagger UI visualisiert die API Endpoints, ob und welche Parameter für diese benötigt werden und die entsprechende Rückgabe. Die visuelle Dokumentation soll sowohl die Backend-Implementierung als auch die Nutzung auf der Client-Seite erleichtern. 



