\namedchapter{Fazit \& Ausblick}{T}
Mit dem Abschluss des Projektes hatten wir eine skalierbare und leicht erweiterbare Lösung geschaffen, mit der Statistance ihre Kunden über das Sage-Warenwirtschaftssystem anbinden und all ihre Anforderungen abdecken kann. Da wir für die Lösung eine flexible Microservice-Architektur verwendet haben, kann die Lösung ohne viel Aufwand um weitere Konnektoren und Services erweitert werden. Auf Grund der guten wiederverwendbaren Struktur unseres bereits implementierten sage-connectors, eignet sich dieser dementsprechend sehr gut als Vorlage für neue Konnektoren. Hierzu müsste lediglich das Repository geklont und gegebenenfalls die QuartzJobBeans sowie die Batch Jobs für neue Drittsysteme angepasst werden. \\
Bei steigenden Anforderungen bezüglich Skalierung und Standardisierung, macht die Nutzung und Anbindung an das OIH aus unserer Sicht Sinn, da mehr Funktionalitäten des OIH genutzt werden können und sich dadurch ein besseres Preis-/Nutzen-Verhältnis ergibt. Im Gegensatz zu unserer aktuellen Lösung, in der das gesamte System statisch aufgebaut ist, können Konnektoren (Adapter/Transformer) im OIH dynamisch je nach Bedarf bereitgestellt werden. Dies ermöglicht eine horizontale Skalierung und führt dazu, dass auch höhere Lasten über verschiedene Knoten (Server) auf der Plattform bedient werden können, was in unserer aktuellen Lösung aktuell nicht möglich ist. Zudem können die verschiedenen Integration Components in Integration Flows im OIH besser wiederverwendet und orchestriert werden als in unserer aktuellen Lösung. Das liegt daran, weil in unserem aktuellen System gesamte Flows in einem dedizierten Konnektor implementiert und keine zentralen Services für die Erstellung und Orchestrierung von verschiedenen Flows in unserer aktuellen Lösung verfügbar sind. \\
Dennoch haben wir mit unserer Lösung für Statistance einen wichtige Grundbasis für die weiteren Arbeiten und Integration neuer Drittsysteme geschaffen, die im Prinzip nach gewissen Anpassungen der Konnektoren auch an das OIH angebunden werden könnte. Für die Zukunft würden wir das OIH auf Grund der noch größeren Skalierbarkeit, Flexibilität und Standardisierung empfehlen.
