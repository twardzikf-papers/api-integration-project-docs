Auf Grund der Tatsache, dass das OIH ziemlich umfangreich und komplex ist, hatten wir schlussendlich eine weitere Option in Erwägung gezogen, bei der wir eine ähnliche Skalierung und Erweiterbarkeit wie das OIH angestrebt haben, welche jedoch auch individuelle Anforderungen von Statistance abdecken konnte und weniger komplex sowie umfangreich ist. Um dies zu erreichen, hatten wir hierfür eine Microservice-Architektur mit verschiedenen Services vorgesehen. Die Idee war dabei, dass pro Drittsystem ein separater Konnektor implementiert wird, der die Daten abfragen, transformieren und speichern kann. Diesbezüglich würden die Daten nach der Transformierung aus den Drittsystemen in einer zentralen Datenbank gespeichert werden, um eine schnelle und effiziente Abfrage der Daten über bereitgestellte APIs sicherzustellen. Damit die APIs leicht zu verwalten und einheitlich sichtbar sind, hatten wir zudem ein API Gateway\footnote{Siehe https://microservices.io/patterns/apigateway.html} vorgesehen, welchen das zentrale Routing an die entsprechenden Endpoints der Konnektoren übernimmt. Auf diesem Weg sollten auch die verschiedenen Services des Systems über einen separaten Service zentral konfiguriert werden können, sodass der Aufwand im Betrieb gering gehalten wird. Diese Konfigurationen könnten dann auf externe Repositories (z.B. Consul oder Git Repository) gespeichert werden. Des Weiteren wäre das System für zukünftige Anforderungen gut erweiterbar und neue zusätzliche Services könnten ohne viel Aufwand hinzugefügt werden. Dabei war es wichtig, dass die Applikation auch auf traditioneller Weise On-Premise ohne Cloud-Technologien beim Kunden ohne viel Aufwand betrieben werden kann. Die Details zu der Architektur werden intensiv in Kapitel \ref{chap:Softwarearchitektur API-Design} behandelt. \\
Bei dieser Lösung ergaben sich einige Vorteile für Statistance, da hierbei sehr gut auf individuelle Anforderungen von Statistance eingegangen werden konnte und nur die Services entwickelt werden, die tatsächlich benötigt werden. Zudem war die Lösung flexibel gestaltbar, sehr gut skalierbar und erweiterbar für neue Drittsysteme, da Konnektoren unabhängig voneinander sind und als autarke Services fungieren. Eine Wiederverwendbarkeit der Konnektoren beziehungsweise Services für neue Kunden war demzufolge gewährleistet und die verschiedenen Systeme und APIs hätten zentral verwaltet werden können. \\
Auf der anderen Seite entstand hierbei auch ein hoher Entwicklungs- und Wartungsaufwand, da auf eine komplette Individualsoftwareentwicklung zurückgegriffen und eine skalierbare Architektur mit verschiedenen Services angestrebt wurde. Im Gegensatz zu der vorherigen Option mit der Nutzung des OIH (siehe Kapitel \ref{subsection:oih}) standen hierbei keine vorhandenen Konnektoren von einer Community zur Verfügung. Zudem waren für die Lösung weniger Features vorgesehen als das OIH.