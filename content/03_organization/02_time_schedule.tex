\section{Zeitplan und Meilensteine}

In der Vorplanungsphase wurden organisatorische Aufgaben abgearbeitet. Diese umfassten die Organisation im Team, wie beispielsweise die Rollenverteilung sowie die konkrete Anforderungsanalyse durch Gespräche mit Statistance. Auch Entwicklungswerkzeuge wurden in der ersten Woche festgelegt. Von Ende Oktober bis Mitte November gab es eine intensive Recherchephase. In dieser Phase wurden verschiedenste Frameworks, Programmiersprachen, Datenbankmodelle und mögliche Architekturen analysiert und auf ihre Eignung hinsichtlich des Projektes evaluiert. In dem darauffolgenden Sprint wurde insbesondere Open Integration Hub als mögliches Framework zur Umsetzung der Aufgabe analysiert und eine Kostenkalkulation aufgestellt. Mögliche Umsetzungsoptionen wurden Statistance präsentiert. Eine konkrete Entscheidungsfindung gab es im Dezember 2019. Hierbei wurde die finale Architektur festgelegt sowie die Designs des Domänenmodells und des Daten-Mappings. Im gleichen Monat begann zusätzlich die konkrete Implementierung des zu entwickelnden Architekturmodells. Eine Zwischenpräsentation lieferte einen Überblick über das bis dato Geschehene. Während im Dezember 2019 zunächst die Integration von Sage 100 fokussiert wurde, kamen im Januar 2020 die Umsetzung der API, Spring Security und die Implementierung der Batch-Jobs hinzu. Abschließend wurden im Februar 2020 ein Frontend zur Steuerung der Batch-Jobs und das API Gateway implementiert sowie das Config Management realisiert. Insgesamt gab es drei Prototypen. Die ersten zwei Prototypen waren statisch. Batchjobs wurden im gleichen Turnus ausgeführt. Der dritte Prototyp ist dynamisch, da eine individuelle Ausführung verschiedener Batch-Jobs ermöglicht wird. Der dritte Prototyp stellt zugleich das Endprodukt dar. Tabelle \ref{tab:sprints} bietet einen Überblick über die einzelnen Sprints mit dem entsprechenden Zeitraum als auch über die Kernaufgaben des Sprints. Die zu Beginn und auch im Laufe des Projektes immer wieder angepassten Aufgaben konnten im festgelegten Zeitraum umgesetzt werden. Den Abschluss des Projektes stellte die Abschlusspräsentation am 10. Februar 2020 dar. 

\begin{table}[h!]
\begin{tabular}{|c|c|p{7cm}|c|}
\hline
\textbf{Sprint} & \textbf{Zeitraum} & \textbf{Aufgaben} &  \textbf{Prototyp}\\ \hline \bottomrule
1 & Oktober 2019  & Treffen mit Statistance, Besprechung der konkreten Aufgabe, Team-Organisation &  \\ \hline
2 & Oktober- November 2019  & Recherche (Frameworks, Programmiersprachen, Datenbanken, Schnittstellen zu ERP-Systemen) &  \\ \hline
3 & November - Dezember 2019 & Aufstellung verschiedener Umsetzungsoptionen, Fokus auf Open Integration Hub und Aufstellung der Kostenkalkulation, Anfertigung der Zwischenpräsentation &  \\ \hline
\multicolumn{4}{|c|}{Zwischenpräsentation}   \\ \hline
4 & Dezember 2019 - Januar 2020 & Entscheidung für eine Umsetzungsoption, Design des Domänenmodells und des Mappings der Daten,  Implementierungsstart (Batch-Jobs, Integration Sage 100) & I (statisch) \\ \hline
5 &  Januar 2020 &  Implementierung (Batch-Jobs, API, Spring Security), Set Up auf Testserver von Statistance einrichten & II (statisch) \\ \hline
6 &  Februar 2020 &   Implementierung (Frontend, API Gateway, Config Management), Anfertigung der Abschlusspräsentation & III (dynamisch)  \\
\hline
\multicolumn{4}{|c|}{Abschlusspräsentation}   \\ \hline
\end{tabular}
\caption{Sprints}
\label{tab:sprints}
\end{table}
\newpage
